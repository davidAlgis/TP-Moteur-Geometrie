% !TEX options=-synctex=-1
\documentclass[a4paper,12pt]{article}
\usepackage[utf8]{inputenc}
\usepackage[T1]{fontenc}
\usepackage[french]{babel}
\usepackage{amsmath,amssymb}
\usepackage{xcolor}
\usepackage{geometry}

\geometry{
  a4paper,
  total={170mm,257mm},
  left=20mm,
  top=20mm,
}

\usepackage{extensionCours}

\title{Barème - TP}
\author{}
\date{\today}

\begin{document}

\makeonlytitle

\tipbox{Voilà le détail du barème de chaque TP, sachant que les sous détails du barème ne sont pas exhaustif.}


\section*{Qualité du code et pratiques de développement (7 points)}
\begin{itemize}
    \item \textbf{Git propre} : Commits clairs, atomiques et historique cohérent.
    \item \textbf{Organisation du code} : Code modulaire et bien structuré.
    \item \textbf{Conformité PEP 8} : Respect des conventions Python.
    \item \textbf{Commentaires clairs et précis} : Explications pertinentes.
    \item \textbf{Robustesse du code} : Gestion des erreurs et entrées inattendues.
    \item \textbf{Code testé} : Tests unitaires couvrant divers cas.
\end{itemize}

\section*{Fonctionnalités du TP (5 points)}
\begin{itemize}
    \item \textbf{Implémentation correcte} : Toutes les fonctionnalités demandées sont présentes.
    \item \textbf{Fonctionnement} : Les fonctionnalités fonctionnent comme prévu.
\end{itemize}

\section*{Rapport et mathématiques (8 points)}
\begin{itemize}
    \item \textbf{Clarté et structure du rapport} : Rapport rédigé avec une structure clair et scientifique.
    \item \textbf{Précision des mathématiques} : Mathématiques acceptable, rigoureuse et correctement écrit.
    \item \textbf{Rédaction} : Rapport rédigé dans un français convenable.
    \item \textbf{Analyse et interprétation} : Résultats bien analysés et interprétés.
\end{itemize}

\end{document}
