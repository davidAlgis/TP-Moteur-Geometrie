% !TEX options=-synctex=-1
\documentclass[a4paper,12pt]{article}
\usepackage[utf8]{inputenc}
\usepackage[T1]{fontenc}
\usepackage[french]{babel}
\usepackage{geometry}

\geometry{
  a4paper,
  total={170mm,257mm},
  left=20mm,
  top=20mm,
}

\usepackage{extensionCours}


\title{TP0 - Mise en place}
\author{}
\date{}

\begin{document}

\makeonlytitle

\sectionExercice{L’environnement de développement}

Tout d'abord, on va commencer par installer et configurer les différents logiciels qu'on utilisera aux cours du semestre

\subsection{Installation}

\begin{enumerate}
    \item Installez Git pour le versionnage : \href{https://git-scm.com/downloads}{Git Downloads} (Windows, macOS, Linux)
    \item Pour s'occuper du dépôt Git, je conseille d'utiliser \href{https://www.sublimemerge.com/}{Sublime Merge}, mais c'est à votre convenance si vous préférez utiliser les lignes de commandes ou un autre logiciel.
    \item Installez \href{https://www.sublimetext.com/}{Sublime Text} pour l'IDE. 
    \item Installez \href{https://www.python.org/downloads/}{Python} pour le code. Assurez-vous lors de son installation de bien l'ajouter au \texttt{\$PATH}.
    \item Choisir un nom de projet et un nom de moteur. 
    \item Définissez un chef d'équipe. Celui-ci va créer les git sur \href{https://github.com/}{Github}. 
\end{enumerate}

\subsection{Mise en place du git (pour le chef d'équipe)}

Le chef d'équipe doit mettre en place cette structure
\begin{lstlisting}[language=json]
nom_projet/
|--- rapports-USSI6V/
|--- nom_moteur_librairie/
|--- nom_moteur_executables/
\end{lstlisting}

Pour ça il doit créer les répertoires git suivant en privé : 
\begin{enumerate}
    \item \texttt{nom\_projet} 
    \item \texttt{rapports\_USSI6V}
    \item \texttt{nom\_moteur\_librairie}
    \item \texttt{nom\_moteur\_executables}
\end{enumerate}

Pour le répertoire \texttt{nom\_projet} doit uniquement contenir les trois autres répertoires comme des sous modules, pour ça vous pouvez utiliser la commande \href{https://git-scm.com/book/en/v2/Git-Tools-Submodules}{git submodule add}.

Je vous recommande de créer un fichier \texttt{.gitignore} pour chaque répertoire, celui-ci définit ce qui ne doit pas être poussé sur le serveur.  

\warningbox{Avant de versionner et d'envoyer un fichier sur le serveur git, assurez-vous bien que le fichier \textbf{est nécessaire à la bonne utilisation du projet}. Par exemple, il ne faut jamais versionner un fichier temporaire ou un fichier qui peut être généré automatiquement à partir du reste du code.}

\subsection{Mise en place Sublime Text}

\subsubsection{Installation des packages}

\tipbox{Toute la configuration personnelle de Sublime Text est accessible en cliquant sur \texttt{Preferences/Browse Packages...}. Aussi, une seule personne de l'équipe peut faire cette mise en place et uniquement partager ce dossier aux autres membres de son équipe à la fin.}

Sans package, sublime text est à peine mieux qu'un bloc note. Afin de pouvoir profiter de sa richesse, je vous invite dans un premier temps à installer le \texttt{Package Control}, c'est un gestionnaire de package qui simplifie l'installation des packages, pour ça il suffit de faire:

\begin{enumerate}
    \item Ouvrir la palette de commandes : Win/Linux : \texttt{ctrl+shift+p}, Mac : \texttt{cmd+shift+p}
    \item Tapez \texttt{Install Package Control}, appuyez sur entrée
    \item Ensuite pour installer un package il faut rouvrir la palette de commandes.
    \item Tapez \texttt{Install Package}, appuyez sur entrée
    \item Tapez le nom du package que vous souhaitez puis re-appuyer sur entrée pour l'installer.
    \item Voilà.
\end{enumerate}    


Ensuite, je vous conseille fortement d'installer au moins les packages suivants :
\begin{itemize}
    \item \texttt{LSP} ou \textbf{L}anguage \textbf{S}erver \textbf{P}rotocol. C'est une spécification qui permet de communiquer entre un IDE et des outils pour faciliter le développement de chaque langage.
    \item \texttt{LSP-pylsp} c'est le serveur LSP pour python. 
\end{itemize}


\tipbox{Je recommande fortement à l'un d'entre vous d'aller lire rapidement la documentation de \href{https://lsp.sublimetext.io/language_servers/}{LSP}, mais surtout le \textit{readme} de \href{https://github.com/sublimelsp/LSP-pylsp}{pylsp} afin de configurer vos serveurs. En effet, vous pouvez les configurez dans \texttt{Preferences/Packages Settings/LSP/Servers/LSP-pylsp} en appliquant du reformatage automatique ce qui vous sera \textbf{extrêmement utile} pour la suite du semestre.}   

\subsubsection{Packages facultatifs}

Vous pouvez aussi, pour votre confort, installer les packages suivants :
\begin{itemize}
    \item \texttt{SublimeLinter} permet de mettre en surbrillance dans l'éditeur des parties du code erronées ou problématique.
    \item \texttt{SublimeLinter-flake8} Un ajout à \texttt{SublimeLinter} spécifique pour le python.
    \item \texttt{Terminus} ça installe un terminal dans sublime text qui peut être utilisé pour lancer les \textit{build} python et qui peut être configuré et plus beau.
    \item \texttt{A File Icon} ça permet d'avoir les icônes des différents fichiers en mode projet.
    \item \texttt{TOML} ça permet d'avoir de la coloration syntaxique pour les fichiers \texttt{.toml}.
\end{itemize}

\tipbox{Je vous invite à regarder \href{https://docs.sublimetext.info/index.html}{la documentation non officielle de Sublime text} si vous voulez plus de détail sur le fonctionnement et l'utilisation de celui-ci.}

\subsubsection{Configuration projet}

Maintenant, que vous avez installé les packages, il faut configurer le projet pour sublime text, pour ça je vous invite à créer un nouveau \texttt{sublime-project} en appuyant sur \texttt{Project/Save Project As...} puis sauvegarder le dans \texttt{nom\_projet} ensuite pour le lier à ce dossier vous devez cliquer sur \texttt{Project/Add Folder To Project...}.

Le \texttt{sublime-project} en plus de nous donner une vision globale du projet nous permet de donner des paramètres sublime text propre au projet. Par exemple, si on a tout un système de tests qui est propre à un projet de RTS qu'on développe, on voudrait que celui-ci ne pollue pas tout nos autres projets sublime text.    

\tipbox{Un projet sublime contient un fichier \texttt{sublime-project} qui doit contenir les paramètres et les systèmes de construction propre à votre projet et un fichier \texttt{sublime-workspace} qui contient des informations sur l'état de l'espace de travail du projet, y compris les fichiers ouverts, les configurations de projet, et les paramètres d'affichage pour Sublime Text. Ce dernier est typiquement l'exemple type d'un fichier qui est propre à chaque utilisateur et qui ne doit pas être versionné.}

\sectionExercice{Hello world}

\begin{enumerate}
     \item Créer un fichier \texttt{tp\_0\_ins.py} qui affiche « \textit{Hello World} » quand on l’exécute.
     \item Créer un système de construction pour pouvoir lancer automatiquement \texttt{tp\_0\_ins.py} depuis n'importe quelle endroit dans sublime text et éviter de passer par le terminal. Pour ça vous pouvez vous inspirez de l'exemple de fichier \texttt{sublime-project} de l'appendice \ref{sec:fichier_sublime_project}.
     \item Appuyez sur \texttt{Ctrl + Shift + B} puis appuyez sur entrée à la ligne ayant le nom de votre système de build.
     \item Si ça affiche \texttt{Hello World} vous avez réussi. Bravo !
 \end{enumerate} 


\newpage
\appendix

\section{Fichier sublime project}
\label{sec:fichier_sublime_project}

Voici un exemple de fichier \texttt{sublime-project} :
\begin{lstlisting}[language=json]
{
    "folders": [
        {
            "path": "."
        }
    ],
    "build_systems": [
        {
            // nom de votre systeme de build
            "name": "Nom de votre build",
            // si vous avez ajouter le plugin terminus vous pouvez utiliser cette extension
            "target": "terminus_exec",
            // pour imposer a sublime text de mettre le focus sur la sortie console 
            "focus": true,
            // Pour permettre a sublime text d'identifier les lignes avec des erreurs
            "file_regex": "^([^:]+):([0-9]+):([0-9]+)?.*$",
            // Definie la commande d'annulation quand on fait ctrl+c
            "cancel": "terminus_cancel_build",
            // On distingue linux et windows 
            // en general sur linux python utilise python3 et python pour window
            "linux": {
                // on lance le tp_0_env.py
                "shell_cmd": "python3 \"${project_path:${folder}}/nom_projet/tp_0_env.py\""
            },
            "windows": {
                // on lance le tp_0_env.py
                "shell_cmd": "python \"${project_path:${folder}}/nom_projet/tp_0_env.py\""
            },
            "working_dir": "${project_path:${folder}}/nom_projet",
        },
    ]
}
\end{lstlisting}

\end{document}
