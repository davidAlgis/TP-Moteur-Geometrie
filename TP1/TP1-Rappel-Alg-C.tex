% !TEX options=-synctex=-1
\documentclass[a4paper,12pt]{article}
\usepackage[utf8]{inputenc}
\usepackage[T1]{fontenc}
\usepackage[french]{babel}
\usepackage{amsmath,amssymb}
\usepackage{xcolor}
\usepackage{geometry}

\geometry{
  a4paper,
  total={170mm,257mm},
  left=20mm,
  top=20mm,
}

\usepackage{../extensionCours}

\title{TP1 - Algèbre Linéaire et Plan Complexe}
\author{}
\date{\today}

\begin{document}

\makeonlytitle


\warningbox{Dans ce TP et les suivants, à moins que je vous le précise clairement, il vous est interdit d'utiliser des packages externes !}

\sectionExercice{Algèbre Linéaire}

\begin{enumerate}
    \item Résolvez le système linéaire suivant :
\[
\begin{cases}
2x + 2y = 4,\\[1mm]
3x - 8y  = -1 
\end{cases}
\]
    \item Dans \(\mathbb{R}^2\), considérez la base canonique \(\{e_1, e_2\}\) et une nouvelle base \(\{v_1, v_2\}\) définie par
\[
v_1 = \begin{pmatrix} 1 \\ 1 \end{pmatrix}, \quad
v_2 = \begin{pmatrix} 1 \\ -1 \end{pmatrix}.
\]
Exprimez le vecteur 
\[
w = \begin{pmatrix} 2 \\ 3 \end{pmatrix}
\]
dans la nouvelle base.
    \item Calculer le produit scalaire des vecteurs $v_1$ et $v_2$ définis ci-dessus.
    \item Calculer le déterminant de la matrice:
\[
v_1 = 
    \begin{pmatrix} 
        1 & 2 \\ 
        2 & 3 
    \end{pmatrix}
\]
    \item Programmez les classes suivantes :
    \begin{itemize}
        \item Une classe point ou vecteur $\mathbb{R}^2$.
        \item Une classe matrice $2\times2$.
    \end{itemize}
    Vos classes doivent « pouvoir résoudre » chaque question ci-dessus.
\end{enumerate}

\tipbox{Vous allez écrire les classes mathématiques de bases dont vous allez vous servir pendant tout le reste du semestre, donc assurez-vous d'avoir du code simple, robuste, efficace et pratique pour l'avenir !}


\sectionExercice{Plans Complexes}

Dans cet exercice, vous allez explorer l'univers des fractales en générant et affichant l'ensemble de Julia associé à une valeur de paramètre complexe donnée. 


\subsection{Contexte :}
Pour un nombre complexe fixé \( c \), on définit la suite \(\{z_n\}\) par la relation de récurrence
\[
z_{n+1} = z_n^2 + c,
\]
où \(z_0\) varie dans le plan complexe. L'ensemble de Julia correspondant est la frontière entre l'ensemble des points pour lesquels la suite \(\{z_n\}\) reste bornée et ceux pour lesquels elle diverge.

\bigskip

\subsection{Objectif :}
Réalisez un programme Python qui affiche l'ensemble de Julia d'un nombre complexe $c\in \mathbb{C}$ sur un domaine donné, la couleur de chaque point du domaine doit être :
\begin{enumerate}
    \item noire si la suite $z_n$ est bornée.
    \item varie selon le nombre d'itérations avant divergence.
\end{enumerate}

Pour l'affichage, vous pouvez utiliser ma classe \texttt{renderer.py} qui utilise le package \texttt{pygame}.

\tipbox{Notez que ce n'est pas très pratique d'utiliser les coordonnées « \textit{screen} » de \texttt{pygame}, peut-être est-ce le moment d'utiliser votre classe matrice $2\times2$ pour « changer de base de coordonnées ».}


\end{document}
