\documentclass[a4paper,12pt]{article}
\usepackage[utf8]{inputenc}
\usepackage[T1]{fontenc}
\usepackage[french]{babel}
\usepackage{amsmath,amssymb}
\usepackage{xcolor}
\usepackage{geometry}

\geometry{
  a4paper,
  total={170mm,257mm},
  left=20mm,
  top=20mm,
}

\usepackage{extensionAlgisColor}
% contient lstlisting en python
\usepackage{extensionCours}

\title{Guide des bonnes pratiques pour projet en python}
\author{}
\date{}

\begin{document}

\maketitle

\section{Architecture}

L'organisation d'un projet Python est une étape clé pour garantir sa maintenabilité et son évolutivité. 
Voici quelques recommandations pour structurer vos projets :

\begin{itemize}
    \item \textbf{Organisation simple des dossiers} : Évitez de multiplier les sous-dossiers inutiles. Une organisation claire avec un dossier principal pour le code source et un autre pour les tests permet de garder une structure lisible.
    \item \textbf{Séparation des responsabilités} : Regroupez les fichiers et modules par fonctionnalités. Par exemple, placez les fonctions de manipulation de données dans un module distinct de celui gérant l'interface utilisateur.
    \item \textbf{Nomination cohérente} : Utilisez des noms explicites pour vos fichiers et dossiers. Les noms en minuscules et séparés par des underscores (ex : \texttt{mon\_module.py}) sont une bonne pratique en Python.
\end{itemize}

\section{Notion de module en Python}

En Python, un \emph{module} est simplement un fichier contenant du code Python (fonctions, classes, variables, etc.) que l'on peut importer dans d'autres scripts. 

\begin{itemize}
    \item Pour transformer un dossier en module ou package, il suffit de placer un fichier nommé \texttt{\_\_init\_\_.py} dans ce dossier. Ce fichier peut être vide, mais il indique à Python que le dossier doit être traité comme un package.
    \item L'utilisation de modules permet de séparer le code en parties logiques et réutilisables. Par exemple, vous pouvez avoir un module pour la gestion des entrées/sorties, un autre pour les calculs, et ainsi de suite.
    \item Lors de l'importation, il est recommandé d'utiliser une syntaxe claire comme \texttt{from monpackage import monmodule} ou \texttt{import monpackage.monmodule as mm} pour faciliter la lisibilité et la maintenance du code.
\end{itemize}

Cette approche modulaire rend votre projet plus structuré et facilite les tests unitaires ainsi que la collaboration en équipe.
\section{pyproject.toml}

Utilisez le fichier \texttt{pyproject.toml} pour centraliser la configuration de vos outils. Voici un exemple de \texttt{pyproject.toml} :

\begin{lstlisting}
[build-system]
requires = ["setuptools>=42", "wheel"]
build-backend = "setuptools.build_meta"


[project]
name = "Mon Incroyable Nom de Projet"
version = "0.0.0"
description = "Ma Chouette Description"
authors = [
    { name = "Mon Magnifique Nom", email = "ma-super@addresse.fr" }
]

[tool.flake8]
max-line-length = 79
exclude = ["tests/*"]

# Dependances externes autorisees dans le projet.
[tool.project]
dependencies = [
    "pygame>=2.0.0",
]
\end{lstlisting}

En adoptant cette configuration, vous centralisez vos réglages et facilitez la collaboration sur votre projet.


\section{Dépendances externes}

Pour que l'installation des dépendances externes soit simple et reproductible, suivez ces recommandations :

\begin{itemize}
    \item \textbf{Utilisez un environnement virtuel} : Créez un environnement isolé pour éviter les conflits avec les packages installés globalement. Par exemple, exécutez :
\end{itemize}

\begin{lstlisting}
python -m venv env
env\Scripts\activate
\end{lstlisting}

\begin{itemize}
    \item \textbf{Installez les dépendances via \texttt{pyproject.toml}} : Une fois l'environnement activé, installez les modules requis à l'aide d'un gestionnaire de dépendances ou directement via pip.
\end{itemize}

\begin{lstlisting}
pip install .
\end{lstlisting}

Cela garantit que tout utilisateur disposant du code et du fichier \texttt{pyproject.toml} peut reproduire l'environnement de développement.

\section{Readme}

Le fichier \texttt{README.md} est la vitrine de votre projet. Il doit fournir aux utilisateurs et aux contributeurs toutes les informations nécessaires pour comprendre, installer et utiliser le projet. 

\section{Tests unitaires}

\textbf{On insistera jamais assez implémenter absolument des tests unitaires !} Chaque morceau de code doit être testé et validé par des tests unitaires. De préférence, je vous conseille même de programmer le test avant la fonctionnalité.


En pratique pour les tests, vous pouvez utiliser \texttt{pytest}, il suffit de placer vos tests dans un dossier \texttt{tests/} et nommez vos fichiers avec le préfixe \texttt{test\_}. 

\textbf{Exemple de test avec \texttt{pytest}} :

\begin{lstlisting}[language=python]
def addition(a, b):
    return a + b

def test_addition():
    # Cas normal
    assert addition(2, 3) == 5
    # Cas avec zero
    assert addition(0, 0) == 0
    # Cas negatif
    assert addition(-1, 1) == 0
\end{lstlisting}

Ensuite pour lancer vos tests il suffit d'utiliser la commande \texttt{pytest}.

\section{Ligne de vie du développement}

Pour ce projet, je vous conseille de développer de la façon suivante :

\begin{enumerate}
  \item Formaliser au brouillon ce que l'on souhaite programmer
  \item Ecrire un rapport en latex sur l'organisation de notre code et le pseudo code pour les différents algorithmes (vous pouvez utiliser les packages \texttt{algorithm} et \texttt{algpseudocode} pour rédiger du pseudo code)
  \item Implémenter les tests 
  \item Implémenter les fonctionnalités
  \item S'assurer que les fonctionnalités vérifient les tests
  \item Écrire ou générer la documentation associer à chaque fonctionnalité
\end{enumerate}


\end{document}
