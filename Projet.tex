\documentclass[a4paper,12pt]{article}
\usepackage[utf8]{inputenc}
\usepackage[T1]{fontenc}
\usepackage[french]{babel}
\usepackage{amsmath,amssymb}
\usepackage{xcolor}
\usepackage{geometry}

\geometry{
  a4paper,
  total={170mm,257mm},
  left=20mm,
  top=20mm,
}

\usepackage{../TexTools/extensionAlgisColor}
% contient lstlisting en python
\usepackage{../TexTools/extensionCours}

\title{Guide des bonnes pratiques pour projet en python}
\author{}
\date{}

\begin{document}

\maketitle

\section{Architecture}

L'organisation d'un projet Python est une étape clé pour garantir sa maintenabilité et son évolutivité. 
Voici quelques recommandations pour structurer vos projets :

\begin{itemize}
    \item \textbf{Organisation simple des dossiers} : Évitez de multiplier les sous-dossiers inutiles. Une organisation claire avec un dossier principal pour le code source et un autre pour les tests permet de garder une structure lisible.
    \item \textbf{Séparation des responsabilités} : Regroupez les fichiers et modules par fonctionnalités. Par exemple, placez les fonctions de manipulation de données dans un module distinct de celui gérant l'interface utilisateur.
    \item \textbf{Nomination cohérente} : Utilisez des noms explicites pour vos fichiers et dossiers. Les noms en minuscules et séparés par des underscores (ex : \texttt{mon\_module.py}) sont une bonne pratique en Python.
\end{itemize}

\section{Notion de module en Python}

En Python, un \emph{module} est simplement un fichier contenant du code Python (fonctions, classes, variables, etc.) que l'on peut importer dans d'autres scripts. 

\begin{itemize}
    \item Pour transformer un dossier en module ou package, il suffit de placer un fichier nommé \texttt{\_\_init\_\_.py} dans ce dossier. Ce fichier peut être vide, mais il indique à Python que le dossier doit être traité comme un package.
    \item L'utilisation de modules permet de séparer le code en parties logiques et réutilisables. Par exemple, vous pouvez avoir un module pour la gestion des entrées/sorties, un autre pour les calculs, et ainsi de suite.
    \item Lors de l'importation, il est recommandé d'utiliser une syntaxe claire comme \texttt{from monpackage import monmodule} ou \texttt{import monpackage.monmodule as mm} pour faciliter la lisibilité et la maintenance du code.
\end{itemize}

Cette approche modulaire rend votre projet plus structuré et facilite les tests unitaires ainsi que la collaboration en équipe.
\section{pyproject.toml}

Utilisez le fichier \texttt{pyproject.toml} pour centraliser la configuration de vos outils. Configurez-le de la manière suivante :

\begin{itemize}
    \item \textbf{Gérez les dépendances} : Listez les packages requis et leurs versions pour éviter les conflits.
    \item \textbf{Configurez les outils complémentaires} : Intégrez les configurations pour des outils comme\texttt{flake8} (linter) afin de standardiser votre code.
\end{itemize}

\textbf{Exemple de \texttt{pyproject.toml}} :

\begin{lstlisting}
[build-system]
requires = ["setuptools>=42", "wheel"]
build-backend = "setuptools.build_meta"


[project]
name = "Mon Incroyable Nom de Projet"
version = "0.0.0"
description = "Ma Chouette Description"
authors = [
    { name = "Mon Magnifique Nom", email = "ma-super@addresse.fr" }
]

[tool.flake8]
max-line-length = 79
exclude = ["tests/*"]

# Dépendances externes autorisées dans le projet.
[tool.project]
dependencies = [
    "pygame>=2.0.0",
]
\end{lstlisting}

En adoptant cette configuration, vous centralisez vos réglages et facilitez la collaboration sur votre projet.


\section{Dépendances externes}

% Explique qu'un utilisateur doit pouvoir installer les modules externes facilement avec le pyproject donne la commande
% Explique qu'une bonne pratique est d'utiliser le virtual environnement pour s'assurer que tout fonctionne correctement en dehors de son environement

\section{Readme}

% Expliquer l'importances du readme et le fait qu'il doit détailler une installation et une utilisation exhaustive du projet

\section{Tests unitaires}

Intégrez des tests unitaires dès le début pour valider le fonctionnement de chaque composant de votre code. Suivez ces instructions :

\begin{itemize}
    \item \textbf{Utilisez \texttt{pytest}} : Installez et configurez \texttt{pytest} pour lancer vos tests facilement.
    \item \textbf{Structurez vos tests} : Placez-les dans un dossier \texttt{tests/} et nommez vos fichiers avec le préfixe \texttt{test\_}.
    \item \textbf{Écrivez des tests pour chaque fonction critique} : Couvrez les cas normaux, les erreurs potentielles et les cas limites.
\end{itemize}

\textbf{Exemple de test avec \texttt{pytest}} :

\begin{lstlisting}[language=python]
def addition(a, b):
    return a + b

def test_addition():
    # Cas normal
    assert addition(2, 3) == 5
    # Cas avec zero
    assert addition(0, 0) == 0
    # Cas negatif
    assert addition(-1, 1) == 0
\end{lstlisting}

\textbf{Automatisez l'exécution} : Intégrez \texttt{pytest} dans votre pipeline d'intégration continue (CI) pour lancer les tests à chaque commit et garantir la stabilité du code.

En appliquant ces directives, vous assurerez la qualité et la robustesse de vos projets Python.

\end{document}
