% !TEX options=-synctex=-1
\documentclass[a4paper,12pt]{article}
\usepackage[utf8]{inputenc}
\usepackage[T1]{fontenc}
\usepackage[french]{babel}
\usepackage{geometry}

\geometry{
  a4paper,
  total={170mm,257mm},
  left=20mm,
  top=20mm,
}

\usepackage{extensionCours}

\title{TP0 Bis - Mise en place}
\author{}
\date{}

\begin{document}

\makeonlytitle

\sectionExercice{Installation et environnement de développement \LaTeX}

Pour rédiger vos documents en \LaTeX{} sous Windows, il est recommandé d'installer une distribution complète telle que \TeX Live. Suivez les étapes ci-dessous :

\subsection{Téléchargement et installation}
\begin{enumerate}
    \item Téléchargez l'installateur de \TeX Live : \href{https://tug.org/texlive/windows.html}{\TeX Live}.
    \item Exécutez l'installateur et suivez les instructions à l'écran.
\end{enumerate}

\subsection{Installation des packages pour Sublime Text}
Pour faciliter la rédaction de vos documents \LaTeX{} avec Sublime Text, installez les packages suivants :
\begin{enumerate}
    \item Ouvrez la palette de commandes (Ctrl+Shift+P sur Windows).
    \item Tapez \texttt{Package Control: Install Package} et validez.
    \item Recherchez et installez \texttt{LaTeXTools} : ce package offre des outils de compilation, de visualisation et d'autocomplétion pour \LaTeX.
    \item Recherchez et installez \texttt{LaTeX-cwl} qui fournit des complétions de commandes \LaTeX.
\end{enumerate}

\sectionExercice{Bonjour Monde en \LaTeX}

Dans cet exercice, vous allez créer et compiler un document simple qui affiche « Bonjour Monde ! ». Suivez les étapes ci-dessous :
\begin{enumerate}
    \item Créez un nouveau fichier nommé \texttt{bonjour.tex} à mettre dans le dossier TP0.
    \item Copiez-collez le contenu suivant dans le fichier :
    \begin{lstlisting}[language=TeX]
\documentclass[a4paper,12pt]{article}
\usepackage[utf8]{inputenc}
\usepackage[T1]{fontenc}
\usepackage[french]{babel}
\title{Bonjour Monde}
\author{}
\date{}

\begin{document}
\maketitle

Bonjour Monde !

\end{document}
    \end{lstlisting}
    \item Enregistrez le fichier.
    \item Compilez le document en utilisant \texttt{LaTeXTools} (par exemple via la commande \texttt{Ctrl+B} ou en sélectionnant le build system LaTeX dans le menu \texttt{Tools > Build System}).
    \item Vérifiez que le PDF généré affiche correctement le texte « Bonjour Monde ! ».
\end{enumerate}

\end{document}
