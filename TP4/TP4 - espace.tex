% !TEX options=-synctex=1 -output-directory=temp
\documentclass[a4paper,12pt]{article}
\usepackage[utf8]{inputenc}
\usepackage[T1]{fontenc}
\usepackage[french]{babel}
\usepackage{amsmath,amssymb,amsthm} 
\usepackage{xcolor}
\usepackage{geometry}
\geometry{
  a4paper,
  total={170mm,257mm},
  left=20mm,
  top=20mm,
}

\usepackage{../extensionCours}

\title{TP 4 - Espace}
\author{}
\date{\today}

\begin{document}

\makeonlytitle


\sectionExercice{Théorie}
\label{exo:theorique}


Soit un objet défini dans son repère local. On vous fournit les transformations suivantes à appliquer dans l'ordre indiqué :

\begin{itemize}
  \item Une translation par le vecteur \((a, b, c) = (3, -2, 5)\).
  \item Une rotation autour de l'axe \(z\) de \(\theta = \frac{\pi}{3}\) (60°).
\end{itemize}

\begin{enumerate}
  \item Écrivez individuellement la matrice de translation \(T\) et la matrice de rotation \(R_z\) en coordonnées homogènes (matrices \(4 \times 4\)).
  \item Calculez la matrice de transformation globale \(M\) en composant ces matrices dans l'ordre approprié.
  \item Appliquez \(M\) à un sommet local 
  \[
  V_{\text{local}} = \begin{pmatrix} 1 \\ 0 \\ 0 \\ 1 \end{pmatrix}
  \]
  pour déterminer sa position dans l'espace monde.
  \item Déterminer le déterminant de la matrice $M$.
  \item On applique à l'objet \(H(1.5)\) une homothétie uniforme de facteur $1.5$. Déterminer la nouvelle matrice $M$ de transformation globale.
  \item Déterminer le déterminant de la matrice $M$, en déduire une interprétation géométrique du déterminant.
  \item Soit les coordonnées monde d'un somme de l'objet :  
  \[
  W = \begin{pmatrix} 4 \\ -4 \\ 5 \\ 1 \end{pmatrix}
  \]
  déterminer ses coordonnées locales.
\end{enumerate}


\sectionExercice{Implémentation 3D}
\label{exo:impl}

Dans la suite, on va essayer de reproduire une version simplifié du fonctionnement d'Unity. 

\begin{enumerate}
  \item Implémenter une classe vecteur et matrice adaptées pour la géométrie dans l'espace et les coordonnées homogènes.
  \item Implémenter une façon de calculer les coordonnées barycentrique d'un point pour un triangle donné. 
  \item Implémenter une classe \texttt{gameobject} qui vous servira de réceptacle pour toutes les informations d'un objet dans votre jeu.
  \item Implémenter une classe \texttt{transform} qui contiendra les informations de géométries de votre \texttt{gameobject}.
  \item Implémenter une classe \texttt{renderer} qui contiendra les informations de rendu de votre \texttt{gameobject}.
  \item Implémenter un pipeline de rendu simplifié comme vu en cours pour afficher un cube.
\end{enumerate}

\sectionExercice{Chute des corps}

Maintenant que nous avons les briques de notre moteur, nous souhaitons construire une simulation physique d'une cube. 
L'idée est d'utiliser les équations de Newton pour modéliser le mouvement d'un objet sous l'effet de la gravité ou d'autres forces, puis d'approximer numériquement ces équations à l'aide d'un schéma d'intégration explicite, en particulier la méthode d'Euler. Considérons la deuxième loi de Newton : 

\begin{equation}
    \label{equa:newton}
    \sum F = m\vec{a},
\end{equation}

Celle-ci nous donne directement une expression de l'accélération $a$ en fonction de la masse de l'objet $m$ et des forces appliquées sur celui-ci $F$. Or on sait que l'accélération est la dérivée de la vitesse, on a donc l'égalité suivante $a = \frac{dv}{dt}$. De plus, on sait depuis le cours sur les développements limités que :

\begin{equation}
    \label{equa:dl}
    \frac{dv}{dt} \simeq \frac{v(t+h)-v(t)}{h}
\end{equation}

pour un $h$ suffisamment petit. Ici on va faire évoluer notre simulation par incrémentation de petit pas de temps $\Delta t$, donc on prends $\Delta t = h$. En combinant les équations \eqref{equa:newton} et \eqref{equa:dl} on obtient :

\begin{align}
\sum F &= m \frac{dv}{dt} \\
\sum F & m \frac{v(t+\Delta t)-v(t)}{\Delta t} \\
v(t+\Delta t) &= v(t) + \frac{\Delta t}{m}(\sum F)
\end{align}

Ce qui nous permet de déterminer directement la vitesse au temps $t+\Delta t$ quand on la possède au temps $t$. 


\begin{enumerate}
    \item En utilisant le même raisonnement que la vitesse ci-dessus, déterminer une expression de la nouvelle position $p(t+\Delta t)$ en fonction de la nouvelle vitesse $v(t+\Delta t)$ et de l'ancienne position $p(t)$.  
    \item Implémenter des fonctions pour mettre à jour la physique pendant votre simulation.
    \item Ajouter un cube $h$ mètres au dessus du sol, appliquer la gravité sur celle-ci. Dans un premier temps, on considère qu'aucune réponse de collision n'est appliqué et que le cube est juste stoppée quand son centre atteint le niveau $0$ de la verticale. 
    \item En théorie un cube nécessite $\sqrt{\frac{2h}{g}}$ secondes, quant est-il dans votre simulation ? que pouvez vous en déduire ?
    \item Maintenant, on souhaite simuler une approximation « naïve » de réponse de collision. Pour ça, on suppose que la collision absorbe la moitié de la vitesse de la cube et que celle-ci est renvoyé dans la direction opposée à son impact. Appliquer ce principe à votre cube qui chute.
    \item Simuler la trajectoire d'une balle cubique de tennis propulsé avec une force de $f$ newton vers l'avant et $g$ vers le haut.
\end{enumerate}

\end{document}
