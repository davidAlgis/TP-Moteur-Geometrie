% !TEX options=-synctex=1 -output-directory=temp
\documentclass[a4paper,12pt]{article}
\usepackage[utf8]{inputenc}
\usepackage[T1]{fontenc}
\usepackage[french]{babel}
\usepackage{amsmath,amssymb,amsthm} 
\usepackage{xcolor}
\usepackage{geometry}
\geometry{
  a4paper,
  total={170mm,257mm},
  left=20mm,
  top=20mm,
}

\usepackage{../extensionCours}

\title{TP 5 - Introduction aux quaternions}
\author{}
\date{\today}

\begin{document}

\makeonlytitle


\sectionExercice{Algèbre des quaternions}
\label{exo:theorique}

\begin{enumerate}
	\item Soit \( q = 1 + 2i + 3j + 4k \).
	      \begin{enumerate}
		      \item Calculez la norme \( \lVert q \rVert \).
		      \item Déterminez le conjugué \( \overline{q} \).
		      \item Vérifiez que \( q \overline{q} = \lVert q \rVert^2 \).
		      \item En déduire l'inverse \( q^{-1} \).
	      \end{enumerate}
	\item Soit \( q_1 = 1 + i + j \) et \( q_2 = 1 - j + k \).
	      \begin{enumerate}
		      \item Calculez \( q_1 q_2 \) et \( q_2 q_1 \).
		      \item Vérifiez que la multiplication des quaternions n’est pas commutative.
	      \end{enumerate}
\end{enumerate}


\sectionExercice{Modification d'angle}

On souhaite effectuer une rotation de $135^\circ$ autour de l’axe $z$. On considère un vecteur initial $\mathbf{p} = (1, 0, 0)$.

\begin{enumerate}
	\item \textbf{Approche par angles d’Euler}

	      On encode cette rotation par les angles d’Euler $(\phi, \theta, \psi) = (0^\circ, 0^\circ, 135^\circ)$ selon l’ordre $ZYX$.
	      \begin{enumerate}
		      \item Écrivez les matrices élémentaires $R_x(\phi)$, $R_y(\theta)$ et $R_z(\psi)$.
		      \item Calculez la matrice de rotation totale $R = R_z R_y R_x$.
		      \item Appliquez cette matrice au vecteur $\mathbf{p}$ pour obtenir $\mathbf{p}_1 = R \mathbf{p}$.
	      \end{enumerate}

	\item \textbf{Approche par quaternion}
	      \begin{enumerate}
		      \item Construisez le quaternion unitaire $q = \left[\cos\left(\frac{135^\circ}{2}\right), \sin\left(\frac{135^\circ}{2}\right)\,\mathbf{k}\right]$.
		      \item Écrivez le vecteur initial comme un quaternion pur $p = [0,\,\mathbf{p}]$.
		      \item Calculez $p' = q p q^{-1}$.
		      \item Identifiez la partie vectorielle de $p'$.
	      \end{enumerate}

	\item \textbf{Comparaison}
	      \begin{enumerate}
		      \item Comparez les résultats $\mathbf{p}_1$ et $\mathbf{p}'$. Sont-ils identiques ?
		      \item Commentez les avantages de l’approche quaternion par rapport à l’approche Euler dans ce cas particulier.
	      \end{enumerate}
\end{enumerate}


\sectionExercice{Implémentation Quaternion}
\label{exo:impl}

\begin{enumerate}
	\item Implémenter une classe quaternion.
	\item Implémenter des fonctions pour manipuler vos objects avec des quaternions.
	\item Faites tourner un cube avec vos fonctions de quaternions.
	\item Afficher un avion dans votre moteur\footnote{Un avion de développeur donc 3 cubes devrait suffire.}. Posez pas de question ça servira pour la prochaine fois. 
\end{enumerate}


\end{document}
