% !TEX options=-synctex=1 -output-directory=temp
\documentclass[a4paper,12pt]{article}
\usepackage[utf8]{inputenc}
\usepackage[T1]{fontenc}
\usepackage[french]{babel}
\usepackage{amsmath,amssymb,amsthm} 
\usepackage{xcolor}
\usepackage{geometry}
\geometry{
  a4paper,
  total={170mm,257mm},
  left=20mm,
  top=20mm,
}

\usepackage{../extensionCours}

\title{TP 6 - Blocage de Cardan et Interpolation}
\author{}
\date{\today}

\begin{document}

\makeonlytitle


\sectionExercice{Référentiel local vs global}

Soit un vecteur \( \mathbf{p} = (1, 0, 0) \). On considère deux rotations successives :
\begin{itemize}
  \item une rotation \( \theta = 90^\circ \) autour de \( \mathbf{z} \) dans le repère \textbf{global},
  \item suivie d’une rotation \( \alpha = 90^\circ \) autour de \( \mathbf{y} \) dans le repère \textbf{local} (lié au vecteur déjà tourné).
\end{itemize}

\begin{enumerate}
          \item Construisez les quaternions \( q_1 \) (rotation autour de \( \mathbf{z} \)) et \( q_2 \) (rotation locale autour de \( \mathbf{y} \)).
          \item Donnez le quaternion global \( q_{\text{final}} \) à appliquer sur \( \mathbf{p} \), en respectant l’ordre des référentiels.
          \item Calculez \( p' = q_{\text{final}}\,p\,q_{\text{final}}^{-1} \).
          \item Donnez les coordonnées finales du vecteur tourné.
          \item Que se passe-t-il si les deux rotations sont appliquées dans le même repère (global ou local uniquement) ?
          \item Que remarque-t-on si l’on inverse l’ordre de multiplication \( q_1 \cdot q_2 \) au lieu de \( q_2 \cdot q_1 \) ?
\end{enumerate}

\sectionExercice{Blocage de cardan}
\label{exo:impl}

\begin{enumerate}
	\item Après avoir positionner deux avions dans votre scène faites une rotation autour de l'axe $z$ (le roulis) avec :
    \begin{enumerate}
        \item les angles d'Eulers.
        \item vos nouvelles fonctions de rotation utilisant les quaternions.        
    \end{enumerate}
    \item Que constatez-vous ? interprétez.
\end{enumerate}


\sectionExercice{Interpolation}

\begin{enumerate}
    \item Implementer le LERP et le SLERP
    \item Après avoir positionner deux avions dans votre scène faites une interpolation de $(0,0,0)$ vers $(0,270,0)$ avec :
    \begin{enumerate}
        \item LERP sur les angles d'Eulers.
        \item SLERP sur les quaternions.        
    \end{enumerate}
    \item Que constatez-vous ? interprétez.
\end{enumerate}

\end{document}
