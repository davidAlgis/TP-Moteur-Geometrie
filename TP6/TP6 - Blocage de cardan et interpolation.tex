% !TEX options=-synctex=1 -output-directory=temp
\documentclass[a4paper,12pt]{article}
\usepackage[utf8]{inputenc}
\usepackage[T1]{fontenc}
\usepackage[french]{babel}
\usepackage{amsmath,amssymb,amsthm} 
\usepackage{xcolor}
\usepackage{geometry}
\geometry{
  a4paper,
  total={170mm,257mm},
  left=20mm,
  top=20mm,
}

\usepackage{../extensionCours}

\title{TP 6 - Blocage de Cardan et Interpolation}
\author{}
\date{\today}

\begin{document}

\makeonlytitle


\sectionExercice{Référentiel local vs global}

Soit un vecteur \( \mathbf{p} = (1, 0, 0) \). On considère deux rotations successives :
\begin{itemize}
  \item une rotation \( \theta = 90^\circ \) autour de \( \mathbf{z} \) dans le repère \textbf{global},
  \item suivie d’une rotation \( \alpha = 90^\circ \) autour de \( \mathbf{y} \) dans le repère \textbf{local} (lié au vecteur déjà tourné).
\end{itemize}

\begin{enumerate}
          \item Construisez les quaternions \( q_1 \) (rotation autour de \( \mathbf{z} \)) et \( q_2 \) (rotation locale autour de \( \mathbf{y} \)).
          \item Donnez le quaternion global \( q_{\text{final}} \) à appliquer sur \( \mathbf{p} \), en respectant l’ordre des référentiels.
          \item Calculez \( p' = q_{\text{final}}\,p\,q_{\text{final}}^{-1} \).
          \item Donnez les coordonnées finales du vecteur tourné.
          \item Que se passe-t-il si les deux rotations sont appliquées dans le même repère (global ou local uniquement) ?
          \item Que remarque-t-on si l’on inverse l’ordre de multiplication \( q_1 \cdot q_2 \) au lieu de \( q_2 \cdot q_1 \) ?
\end{enumerate}

\sectionExercice{Blocage de cardan}
\label{exo:impl}

\begin{enumerate}
	\item Après avoir positionner deux avions dans votre scène faites une rotation autour de l'axe $z$ (le roulis) avec :
    \begin{enumerate}
        \item les angles d'Eulers.
        \item vos nouvelles fonctions de rotation utilisant les quaternions.        
    \end{enumerate}
    \item Que constatez-vous ? interprétez.
\end{enumerate}


\sectionExercice{Interpolation}

\begin{enumerate}
    \item Implementer le LERP et le SLERP
    \item Après avoir positionner deux avions dans votre scène faites une interpolation de $(0,0,0)$ vers $(0,270,0)$ avec :
    \begin{enumerate}
        \item LERP sur les angles d'Eulers.
        \item SLERP sur les quaternions.        
    \end{enumerate}
    \item Que constatez-vous ? interprétez.
\end{enumerate}



\sectionExercice{Chute des corps (bonus)}

Maintenant que nous avons les briques de notre moteur, nous souhaitons construire une simulation physique d'un cube. 
L'idée est d'utiliser les équations de Newton pour modéliser le mouvement d'un objet sous l'effet de la gravité ou d'autres forces, puis d'approximer numériquement ces équations à l'aide d'un schéma d'intégration explicite, en particulier la méthode d'Euler. Considérons la deuxième loi de Newton : 

\begin{equation}
    \label{equa:newton}
    \sum F = m\vec{a},
\end{equation}

Celle-ci nous donne directement une expression de l'accélération $a$ en fonction de la masse de l'objet $m$ et des forces appliquées sur celui-ci $F$. Or, on sait que l'accélération est la dérivée de la vitesse, on a donc l'égalité suivante $a = \frac{dv}{dt}$. De plus, on sait depuis le cours sur les développements limités que :

\begin{equation}
    \label{equa:dl}
    \frac{dv}{dt} \simeq \frac{v(t+\Delta t)-v(t)}{\Delta t}
\end{equation}

pour un $\Delta t$ suffisamment petit. En combinant les équations \eqref{equa:newton} et \eqref{equa:dl} on obtient :

\begin{align}
\sum F &= m \frac{dv}{dt} \\
\sum F &= m \frac{v(t+\Delta t)-v(t)}{\Delta t} \\
v(t+\Delta t) &= v(t) + \frac{\Delta t}{m}\bigl(\sum F\bigr)
\end{align}

Ce qui nous permet de déterminer directement la vitesse au temps $t+\Delta t$ quand on la possède au temps $t$.

\subsection*{Prise en compte de la rotation du cube}

Le cube est un corps rigide : sa position est décrite par un vecteur de translation $\mathbf{p}(t)$ et son orientation par un quaternion unitaire $\mathbf{q}(t)$. Sa vitesse est noté $\mathbf{v}(t)$ et sa vitesse angulaire $\mathbf{\omega}(t)$.   
Pour intégrer la rotation, on part du moment résultant $\boldsymbol{\tau}\in\mathbb{R^3}$ (plus communément appelé \emph{torque} en anglais) appliqué au cube et de sa matrice d'inertie $I\in M_{3\times3}$ exprimée dans l'espace local. La matrice d'inertie d'un cube de masse $m$ et de longueur $l$ est donnée par $I = \frac{ml^2}{6}Id$.

L'algorithme de mise à jour est le suivant :

\begin{enumerate}
  \item Calculer l'accélération angulaire (locale) :
        \[
            \boldsymbol{\alpha}(t)
            = I^{-1}
              \Bigl(\boldsymbol{\tau}(t)
              - \boldsymbol{\omega}(t)\times \bigl(I\,\boldsymbol{\omega}(t)\bigr)\Bigr).
        \]
  \item Mettre à jour la vitesse angulaire :
        \[
            \boldsymbol{\omega}(t+\Delta t)
            = \boldsymbol{\omega}(t) + \Delta t\,\boldsymbol{\alpha}(t).
        \]
  \item Mettre à jour le quaternion d'orientation par intégration d'Euler explicite :
        \[
            \dot{\mathbf{q}}(t)
            = \tfrac12\,
              \bigl(0,\boldsymbol{\omega}(t)\bigr)\,\mathbf{q}(t),\qquad
            \mathbf{q}(t+\Delta t)
            = \mathbf{q}(t) + \Delta t\,\dot{\mathbf{q}}(t).
        \]
  \item Re-normaliser $\mathbf{q}(t+\Delta t)$ pour éviter la dérive numérique :
        \[
            \mathbf{q}(t+\Delta t) \;\leftarrow\;
            \frac{\mathbf{q}(t+\Delta t)}{\lVert\mathbf{q}(t+\Delta t)\rVert}.
        \]
\end{enumerate}

\textbf{Remarque :} Dans un moteur physique temps-réel, on utilisera souvent une matrice d'inertie (ou son inverse) pré-calculée dans l'espace local, puis transformée dans le repère monde à chaque pas de simulation si nécessaire.

\begin{enumerate}
    \item En utilisant le même raisonnement que pour la vitesse ci-dessus, déterminer une expression de la nouvelle position $p(t+\Delta t)$ en fonction de la nouvelle vitesse $v(t+\Delta t)$ et de l'ancienne position $p(t)$.  
    \item Implémenter des fonctions pour mettre à jour la physique pendant votre simulation.
    \item Ajouter un cube $h$ mètres au dessus du sol, appliquer la gravité sur celui-ci. Dans un premier temps, on considère qu'aucune réponse de collision n'est appliquée et que le cube est juste stoppé quand son centre atteint le niveau $0$ de la verticale. 
    \item En théorie un cube nécessite $\sqrt{\tfrac{2h}{g}}$ secondes pour atteindre le sol ; qu'en est-il dans votre simulation ? Que pouvez-vous en déduire ?
    \item Maintenant, on souhaite simuler une approximation \og naïve \fg{} de réponse de collision. Pour cela, on suppose que la collision absorbe la moitié de la vitesse du cube et que celle-ci est renvoyée dans la direction opposée à son impact. Appliquer ce principe à votre cube qui chute.
    \item Simuler la trajectoire d'une balle cubique de tennis propulsée avec une force de $f$ newtons vers l'avant et $g$ vers le haut.
    \item Implémentez la mise à jour de l'orientation dans votre code.
    \item Tester en appliquant un moment résultat sur la verticale du cube. 
    \item Faites des simulations avec des cubes de différentes masses et de différentes longueur. 
\end{enumerate}



\end{document}
