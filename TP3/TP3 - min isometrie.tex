% !TEX options=-synctex=1 -output-directory=temp
\documentclass[a4paper,12pt]{article}
\usepackage[utf8]{inputenc}
\usepackage[T1]{fontenc}
\usepackage[french]{babel}
\usepackage{amsmath,amssymb,amsthm} 
\usepackage{xcolor}
\usepackage{geometry}
\geometry{
  a4paper,
  total={170mm,257mm},
  left=20mm,
  top=20mm,
}

\usepackage{../extensionCours}


\title{TP 3 - Minimisation des isométries}
\author{}
\date{\today}

\begin{document}

\makeonlytitle

\sectionExercice{Exercice Théorique}
\label{exo:theorique}

Vous êtes développeur d'un jeu sous Unity. Malheureusement, comme à son habitude le logiciel est défectueux est l'éditeur ne fonctionne plus du tout\footnote{Si vous pensez que ce scénario est rocambolesque, je vous invite à considérer la version plus probable où vous ne souhaiter pas utiliser l'éditeur, car tout les \textit{gameobjects} sont instanciés dynamiquement à l'initialisation.}, de même que les fonctions qui permettent de définir la position et l'orientation manuellement.  

\begin{enumerate}
  \item Soit un personnage modélisé par un sprite, l'utilisateur souhaite le déplacer à la position $\begin{pmatrix} 1 & 2\end{pmatrix}$, donner la matrice de changement de coordonnée à appliquer au personnage pour lui donner sa nouvelle position.
  \item % question pour modifier l'orientation du personnage
  \item % question où on ajoute un gameobject enfant au personnage par exemple son épais et on souhaite la bouger relativement au personnage...  
\end{enumerate}

\sectionExercice{Implémentation des similitudes}
\label{exo:impl}


\begin{enumerate}
  \item Implémenter une classe vecteur et matrice adaptées pour les coordonnées homogènes.
  \item Implémenter les similitudes.
  \item Implémenter une classe \textit{gameobject} qui permet de localiser et transformer un objet quelconque dans l'espace.
\end{enumerate}
%TODO

\sectionExercice{Minimisation des similitudes}

% TODO TP : implémenter un jeu où l'objectif est de trouver la combinaison la plus faible d'isométrie et d'homothétie pour bouger un triangle.

\end{document}
