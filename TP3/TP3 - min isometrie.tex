% !TEX options=-synctex=1 -output-directory=temp
\documentclass[a4paper,12pt]{article}
\usepackage[utf8]{inputenc}
\usepackage[T1]{fontenc}
\usepackage[french]{babel}
\usepackage{amsmath,amssymb,amsthm} 
\usepackage{xcolor}
\usepackage{geometry}
\geometry{
  a4paper,
  total={170mm,257mm},
  left=20mm,
  top=20mm,
}

\usepackage{../extensionCours}


\title{TP 3 - Minimisation des isométries}
\author{}
\date{\today}

\begin{document}

\makeonlytitle

\sectionExercice{Exercice Théorique}
\label{exo:theorique}

%TODO

\sectionExercice{Implémentation des isométries}
\label{exo:impl}

%TODO

\sectionExercice{Triangulation de Delaunay}

% TODO TP : implémenter un jeu où l'objectif est de trouver la combinaison la plus faible d'isométrie et d'homothétie pour bouger un triangle.

\end{document}
