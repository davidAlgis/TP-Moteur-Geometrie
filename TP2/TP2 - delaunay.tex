% !TEX options=-synctex=1 -output-directory=temp
\documentclass[a4paper,12pt]{article}
\usepackage[utf8]{inputenc}
\usepackage[T1]{fontenc}
\usepackage[french]{babel}
\usepackage{amsmath,amssymb,amsthm} 
\usepackage{xcolor}
\usepackage{geometry}
\geometry{
  a4paper,
  total={170mm,257mm},
  left=20mm,
  top=20mm,
}

\usepackage{../extensionCours}

\newtheorem{theorem}{Théorème}[section]
\theoremstyle{definition}
\newtheorem{definition}{Définition}[section]

\title{TP 2 - Géométrie Classique et Triangulation de Delaunay}
\author{}
\date{\today}

\begin{document}

\makeonlytitle

\section{Rappels de Géométrie Classique}
\label{sec:rappels_de_g_om_trie_classique}
Avant de débuter, il est utile de rappeler quelques résultats fondamentaux de la géométrie :

\begin{theorem}[Théorème de Pythagore]
Soit \( \triangle ABC \) un triangle rectangle en \(C\). Alors, la relation suivante relie les longueurs des côtés :
\[
AB^2 = AC^2 + BC^2.
\]
\end{theorem}

\begin{theorem}[Théorème de Thalès]
Soit \( \triangle ABC \) et une droite parallèle à la base \(BC\) qui coupe les côtés \(AB\) et \(AC\) en \(D\) et \(E\) respectivement. Alors, on a :
\[
\frac{AD}{DB} = \frac{AE}{EC}.
\]
\]
\end{theorem}

\begin{theorem}[Somme des angles]
La somme des angles d’un triangle est égale à \(180^\circ\).
\end{theorem}

\begin{definition}[Rappel de droite particulière]
\begin{itemize}
  \item Une \textbf{médiane} relie un sommet au milieu du côté opposé.
  \item Une \textbf{hauteur} est un segment perpendiculaire à un côté issu du sommet opposé.
  \item Une \textbf{bissectrice} divise un angle en deux angles de même mesure.
\end{itemize}
\end{definition}

% TODO définition du cercle circonscrit.


\sectionExercice{Exercices Théoriques}


% TODO donner du contexte sur une figure et des mesure à déterminer
\begin{enumerate}
  % \item TODO construction à partir des droites décrites ci-dessus pour obtenir les conditions nécessaire à l'utilisation du théorème de Thalès pour la question suivante
  % \item TODO faire une question avec Thalès 
  % \item TODO faire une question avec Pythagore
\end{enumerate}

\sectionExercice{Implémentation d'une Classe \texttt{Triangle}}

\begin{enumerate}
  \item Créer une classe \texttt{Triangle}.
  \item Utiliser cette classe pour vérifier vos résultats de l'énoncé ci-dessus.
\end{enumerate} 

\sectionExercice{Triangulation de Delaunay}

La triangulation de Delaunay d'un ensemble de points dans le plan consiste à diviser le plan en triangles tels que, pour chacun d'eux, le cercle circonscrit ne contient aucun autre point de l'ensemble. Ce procédé présente l'avantage de maximiser le plus petit angle parmi tous les triangles, ce qui permet d'éviter les triangles trop « allongés ».

% TODO expliquer les applications de la triangulation de Delaunay

\subsection{Rappel Théorique}

% TODO expliquer comment construire la triangulation de Delaunay avec Bowyer-Watson, donner le pseudo code avec les packages
% \usepackage{algorithm}
% \usepackage{algpseudocode}
% \algrenewcommand{\algorithmiccomment}[1]{\hfill \textcolor{greenF}{// #1}}



\end{document}
