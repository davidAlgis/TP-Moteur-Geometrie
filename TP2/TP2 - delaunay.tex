% !TEX options=-synctex=1 -output-directory=temp
\documentclass[a4paper,12pt]{article}
\usepackage[utf8]{inputenc}
\usepackage[T1]{fontenc}
\usepackage[french]{babel}
\usepackage{amsmath,amssymb,amsthm} 
\usepackage{xcolor}
\usepackage{geometry}
\geometry{
  a4paper,
  total={170mm,257mm},
  left=20mm,
  top=20mm,
}

\usepackage{../extensionCours}
\usepackage{algorithm}       % Pour l'algorithme en pseudo-code
\usepackage{algpseudocode}   % Pour l'algorithme en pseudo-code

\newtheorem{theorem}{Théorème}[section]
\theoremstyle{definition}
\newtheorem{definition}{Définition}[section]

\title{TP 2 - Géométrie Classique et Triangulation de Delaunay}
\author{}
\date{\today}

\begin{document}

\makeonlytitle

\sectionExercice{Exercice Théorique}
\label{exo:theorique}

Soit les points: $A=\begin{pmatrix}0 & 0\end{pmatrix}$, $B=\begin{pmatrix}4 & 0\end{pmatrix}$, $C=\begin{pmatrix}2 & 2\end{pmatrix}$ dans la repère orthonormée :

\begin{equation*}
\left(\vec{e_1} = \begin{pmatrix}1 & 0\end{pmatrix}, \vec{e_2} = \begin{pmatrix}0 & 1\end{pmatrix}\right).
\end{equation*}

\begin{enumerate}
  \item On définit le cercle $\mathcal{C}_1$ de diamètre le segment $[AB]$, montrer que le point $C$ appartient à $\mathcal{C}_1$, 
  \item En utilisant le théorème de Pythagore montrer que le triangle $ABC$ est rectangle.
  \item Soit $D$ le centre de $\mathcal{C}_1$, soit $\Delta$ la bissectrice de l'angle $\widehat{CDB}$, on note $E$ l'intersection entre $\mathcal{C}_1$ et $\Delta$. En déduire la longueur du segment $[DE]$. 
  \item Soit $\mathcal{C}_2$ le cercle de centre $E$ et de rayon $[DE]$. Par définition $D$ est la première intersection entre $\mathcal{C}_2$ et $\Delta$, on note $F$ la seconde intersection entre $\mathcal{C}_2$ et $\Delta$. Donnez la longueur du segment $[DF]$.
  \item On note $d$ la droite formée par les points $[BE]$, soit $d'$ la droite définie comme l'unique droite parallèle à $d$ et passant par le point $F$. On note $G$ l'intersection entre la droite $d'$ et l'axe engendré par $\vec{e_1}$. En utilisant le théorème de Thalès, déterminez les coordonnées du point $G$.
  \item Représenter la figure obtenue à partir des différentes constructions ci-dessus. Vous pouvez utiliser le package \href{https://www.overleaf.com/learn/latex/TikZ_package}{TikZ} qui permet de générer des figures vectorisées dans le style du LaTex\footnote{Vous pouvez vous aider de l'éditeur en ligne \href{https://www.mathcha.io/}{Mathcha} est un éditeur WYSIWYG qui permet de directement visualiser et de générer du code TikZ.}.  
\end{enumerate}

\sectionExercice{Implémentation d'une Classe \texttt{Triangle}}
\label{exo:triangle}
\begin{enumerate}
  \item Créez une classe \texttt{Triangle} qui stocke les coordonnées des sommets et permet de calculer, par exemple, les longueurs des côtés, le périmètre et l’aire du triangle. 
  \item Utilisez cette classe pour vérifier vos résultats obtenus lors de l'exercice \ref{exo:theorique}.
\end{enumerate} 


\tipbox{Pour l'aspect rendu vous pouvez utiliser les nouvelles fonctions de la classe \texttt{renderer.py}, notamment \texttt{draw_triangle}, \texttt{draw_segment} et \texttt{draw_circle}.}

\sectionExercice{Triangulation de Delaunay}

La triangulation de Delaunay d'un ensemble de points dans le plan vise à diviser ce plan en triangles de telle sorte que pour chaque triangle, le cercle circonscrit ne contient aucun autre point de l'ensemble. Cette propriété a deux avantages majeurs : elle évite la formation de triangles très allongés et elle améliore la qualité du maillage pour des applications telles que la simulation physique, la reconstruction de surfaces ou la navigation robotique.

Le pseudo-code ci-dessous résume l'algorithme :

\begin{algorithm}
\caption{Triangulation de Delaunay par Bowyer-Watson}
\begin{algorithmic}[1]
\State \textbf{Entrée :} Ensemble de points \(P\)
\State \textbf{Sortie :} Triangulation de Delaunay de \(P\)
\State Créer un super-triangle \(T_s\) qui contient tous les points de \(P\)
\State Initialiser la triangulation \(\mathcal{T} \gets \{T_s\}\)
\For{chaque point \(p \in P\)}
    \State \emph{Trouver l'ensemble} \(\mathcal{T}_p\) \emph{des triangles dont le cercle circonscrit contient \(p\)}
    \State \emph{Déterminer le bord de la cavité} formée par \(\mathcal{T}_p\)
    \State Supprimer tous les triangles de \(\mathcal{T}_p\) de \(\mathcal{T}\)
    \For{chaque arête \(e\) du bord}
         \State Créer un nouveau triangle \(t\) avec \(p\) et l'arête \(e\)
         \State Ajouter \(t\) à \(\mathcal{T}\)
    \EndFor
\EndFor
\State Supprimer de \(\mathcal{T}\) tous les triangles contenant un sommet de \(T_s\)
\State \Return \(\mathcal{T}\)
\end{algorithmic}
\end{algorithm}

\begin{enumerate}
  \item Implémentez la triangulation de Delaunay d'un ensemble de points aléatoires.
\end{enumerate}

\end{document}
