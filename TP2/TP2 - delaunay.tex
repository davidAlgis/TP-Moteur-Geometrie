% !TEX options=-synctex=1 -output-directory=temp
\documentclass[a4paper,12pt]{article}
\usepackage[utf8]{inputenc}
\usepackage[T1]{fontenc}
\usepackage[french]{babel}
\usepackage{amsmath,amssymb,amsthm} 
\usepackage{xcolor}
\usepackage{geometry}
\geometry{
  a4paper,
  total={170mm,257mm},
  left=20mm,
  top=20mm,
}

\usepackage{../extensionCours}
\usepackage{algorithm}       % Pour l'algorithme en pseudo-code
\usepackage{algpseudocode}   % Pour l'algorithme en pseudo-code

\newtheorem{theorem}{Théorème}[section]
\theoremstyle{definition}
\newtheorem{definition}{Définition}[section]

\title{TP 2 - Géométrie Classique et Triangulation de Delaunay}
\author{}
\date{\today}

\begin{document}

\makeonlytitle

\sectionExercice{Exercice Théorique}
\label{exo:theorique}

Soit les points: $A=\begin{pmatrix}0 & 0\end{pmatrix}$, $B=\begin{pmatrix}4 & 0\end{pmatrix}$, $C=\begin{pmatrix}2 & 2\end{pmatrix}$ dans la repère orthonormée :

\begin{equation*}
\left(\vec{e_1} = \begin{pmatrix}1 & 0\end{pmatrix}, \vec{e_2} = \begin{pmatrix}0 & 1\end{pmatrix}\right).
\end{equation*}

\begin{enumerate}
  \item On définit le cercle $\mathcal{C}_1$ de diamètre le segment $[AB]$, montrer que le point $C$ appartient à $\mathcal{C}_1$, 
  \item En utilisant le théorème de Pythagore montrer que le triangle $ABC$ est rectangle.
  \item Soit $D$ le centre de $\mathcal{C}_1$, soit $\Delta$ la bissectrice de l'angle $\widehat{CDB}$, on note $E$ l'intersection entre $\mathcal{C}_1$ et $\Delta$. En déduire la longueur du segment $[DE]$. 
  \item Soit $\mathcal{C}_2$ le cercle de centre $E$ et de rayon $[DE]$. Par définition $D$ est la première intersection entre $\mathcal{C}_2$ et $\Delta$, on note $F$ la seconde intersection entre $\mathcal{C}_2$ et $\Delta$. Donnez la longueur du segment $[DF]$.
  \item On note $d$ la droite formée par les points $[BE]$, soit $d'$ la droite définie comme l'unique droite parallèle à $d$ et passant par le point $F$. On note $G$ l'intersection entre la droite $d'$ et l'axe engendré par $\vec{e_1}$. En utilisant le théorème de Thalès, déterminez les coordonnées du point $G$.
  \item Représenter la figure obtenue à partir des différentes constructions ci-dessus. Vous pouvez utiliser le package \href{https://www.overleaf.com/learn/latex/TikZ_package}{TikZ} qui permet de générer des figures vectorisées dans le style du LaTex\footnote{Vous pouvez vous aider de l'éditeur en ligne \href{https://www.mathcha.io/}{Mathcha} est un éditeur WYSIWYG qui permet de directement visualiser et de générer du code TikZ.}.  
\end{enumerate}

\sectionExercice{Implémentation d'une Classe \texttt{Triangle}}
\label{exo:triangle}
\begin{enumerate}
  \item Créez une classe \texttt{Triangle} qui stocke les coordonnées des sommets et permet de calculer, par exemple, les longueurs des côtés, le périmètre et l’aire du triangle.
  \item Utilisez cette classe pour vérifier vos résultats obtenus lors de l'exercice \ref{exo:theorique}.
\end{enumerate} 

\sectionExercice{Triangulation de Delaunay}

La triangulation de Delaunay d'un ensemble de points dans le plan consiste à diviser le plan en triangles tels que, pour chacun d'eux, le cercle circonscrit ne contient aucun autre point de l'ensemble. Ce procédé présente l'avantage de maximiser le plus petit angle parmi tous les triangles, ce qui permet d'éviter des triangles trop « allongés » et d'obtenir ainsi un maillage de meilleure qualité.

\subsection{Applications de la Triangulation de Delaunay}

La triangulation de Delaunay est utilisée dans de nombreux domaines :
\begin{itemize}
  \item \textbf{Maillage numérique et analyse par éléments finis :} Pour générer des réseaux de maillage optimisés dans la modélisation de structures et la simulation physique.
  \item \textbf{Systèmes d'information géographique (SIG) :} Pour l'interpolation de données spatiales et la reconstruction de surfaces (modèles de terrain).
  \item \textbf{Vision par ordinateur :} Pour la reconstruction 3D à partir de nuages de points et la segmentation d'images.
  \item \textbf{Robotique et navigation :} Pour optimiser les trajectoires et la détection d'obstacles dans un environnement donné.
\end{itemize}

\subsection{Rappel Théorique et Algorithme de Bowyer-Watson}

Pour construire la triangulation de Delaunay, on utilise fréquemment l'algorithme de Bowyer-Watson qui procède de la manière suivante :

\begin{enumerate}
  \item \textbf{Création d'un super-triangle} \\
  Définir un triangle suffisamment grand pour englober tous les points de l'ensemble.

  \item \textbf{Insertion itérative des points} \\
  Pour chaque point \(p\) de l'ensemble :
  \begin{enumerate}
    \item Identifier tous les triangles dont le cercle circonscrit contient \(p\).
    \item Déterminer le bord de la cavité formée par ces triangles.
    \item Supprimer ces triangles de la triangulation courante.
    \item Pour chacune des arêtes du bord, créer un nouveau triangle en reliant cette arête à \(p\).
  \end{enumerate}

  \item \textbf{Nettoyage final} \\
  Supprimer tous les triangles qui contiennent un sommet du super-triangle.
\end{enumerate}

Le pseudo-code suivant illustre cet algorithme :

\begin{algorithm}
\caption{Triangulation de Delaunay par Bowyer-Watson}
\begin{algorithmic}[1]
\State \textbf{Entrée :} Ensemble de points \(P\)
\State \textbf{Sortie :} Triangulation de Delaunay de \(P\)
\State Créer un super-triangle \(T_s\) englobant tous les points de \(P\)
\State \(\mathcal{T} \gets \{T_s\}\)
\For{chaque point \(p \in P\)}
    \State \(\mathcal{T}_p \gets \{\}\)
    \For{chaque triangle \(t \in \mathcal{T}\)}
        \If{\(p\) est dans le cercle circonscrit de \(t\)}
            \State Ajouter \(t\) à \(\mathcal{T}_p\)
        \EndIf
    \EndFor
    \State Déterminer le bord de la cavité formée par les triangles de \(\mathcal{T}_p\)
    \State Supprimer les triangles de \(\mathcal{T}_p\) de \(\mathcal{T}\)
    \For{chaque arête \(e\) sur le bord}
         \State Créer un nouveau triangle \(t'\) avec \(p\) et \(e\)
         \State Ajouter \(t'\) à \(\mathcal{T}\)
    \EndFor
\EndFor
\State Supprimer de \(\mathcal{T}\) les triangles contenant un sommet du super-triangle \(T_s\)
\State \Return \(\mathcal{T}\)
\end{algorithmic}
\end{algorithm}

\end{document}
